\documentclass[a4paper,landscape]{article}
\usepackage[german]{babel}
\usepackage[top=0.1in, left=0.1in, right=2.4in, bottom=0.1in]{geometry}
\usepackage{graphicx,color}
\usepackage{ragged2e}
\definecolor{orange}{RGB}{255,97,0}
\definecolor{blue}{RGB}{68,90,111}
\definecolor{hintergrund}{rgb}{0.99,0.99,0.99}
\pagecolor{hintergrund}
\renewcommand{\baselinestretch}{1.2}
\usepackage{fontspec} 
\setmainfont[Mapping=text-tex]{Lato Regular} 
\begin{document}
\thispagestyle{empty}
\vspace*{2cm}
\marginpar{\RaggedRight
\vspace*{-1cm}
Spree, Reinhard: \textit{Wachstumstrends und Konjunkturzyklen in der deutschen Wirtschaft von 1820 bis 1913}. Vandenhoeck \& Ruprecht, Göttingen 1978. GESIS Köln, Deutschland ZA8053 Datenfile Version 1.0.0.\\~\\Spree unternimmt den Versuch, Wachstum und Konjunktur anhand von 18 ausgewählten Indikatoren zu bestimmen. Die hier exemplarisch dargestellten Reihen zeigen die Eheschließungen und den Index der Nahrungsmittelpreise. Wie man deutlich sieht, verläuft die Konjunktur bei den Eheschließungen und den Nahrungsmittelpreisen bis etwa 1867 gegenläufig, erst danach löst sich dieser Zusammenhang auf. Steigende Nahrungsmittelpreise gehen bis 1867 offensichtlich mit einer Verschlechterung der Versorgungslage einher, was viele Menschen dazu veranlasste, nicht zu heiraten. Diese Gegenläufigkeit zwischen Heiratsverhalten und ökonomischer Versorgungslage ist typisch für agrarisch geprägte Volkswirtschaften. Erst im Laufe der Hochindustrialisierung hat sich dieser Zusammenhang aufgelöst.}\begin{picture}(0,485)
\put(-50,-55)
{\includegraphics[scale=0.89]{../pdf/timeseries_with_trend_3x1_inc.pdf}}\end{picture}
\end{document}